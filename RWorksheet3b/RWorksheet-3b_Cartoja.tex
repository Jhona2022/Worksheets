% Options for packages loaded elsewhere
\PassOptionsToPackage{unicode}{hyperref}
\PassOptionsToPackage{hyphens}{url}
%
\documentclass[
]{article}
\usepackage{amsmath,amssymb}
\usepackage{lmodern}
\usepackage{iftex}
\ifPDFTeX
  \usepackage[T1]{fontenc}
  \usepackage[utf8]{inputenc}
  \usepackage{textcomp} % provide euro and other symbols
\else % if luatex or xetex
  \usepackage{unicode-math}
  \defaultfontfeatures{Scale=MatchLowercase}
  \defaultfontfeatures[\rmfamily]{Ligatures=TeX,Scale=1}
\fi
% Use upquote if available, for straight quotes in verbatim environments
\IfFileExists{upquote.sty}{\usepackage{upquote}}{}
\IfFileExists{microtype.sty}{% use microtype if available
  \usepackage[]{microtype}
  \UseMicrotypeSet[protrusion]{basicmath} % disable protrusion for tt fonts
}{}
\makeatletter
\@ifundefined{KOMAClassName}{% if non-KOMA class
  \IfFileExists{parskip.sty}{%
    \usepackage{parskip}
  }{% else
    \setlength{\parindent}{0pt}
    \setlength{\parskip}{6pt plus 2pt minus 1pt}}
}{% if KOMA class
  \KOMAoptions{parskip=half}}
\makeatother
\usepackage{xcolor}
\usepackage[margin=1in]{geometry}
\usepackage{color}
\usepackage{fancyvrb}
\newcommand{\VerbBar}{|}
\newcommand{\VERB}{\Verb[commandchars=\\\{\}]}
\DefineVerbatimEnvironment{Highlighting}{Verbatim}{commandchars=\\\{\}}
% Add ',fontsize=\small' for more characters per line
\usepackage{framed}
\definecolor{shadecolor}{RGB}{248,248,248}
\newenvironment{Shaded}{\begin{snugshade}}{\end{snugshade}}
\newcommand{\AlertTok}[1]{\textcolor[rgb]{0.94,0.16,0.16}{#1}}
\newcommand{\AnnotationTok}[1]{\textcolor[rgb]{0.56,0.35,0.01}{\textbf{\textit{#1}}}}
\newcommand{\AttributeTok}[1]{\textcolor[rgb]{0.77,0.63,0.00}{#1}}
\newcommand{\BaseNTok}[1]{\textcolor[rgb]{0.00,0.00,0.81}{#1}}
\newcommand{\BuiltInTok}[1]{#1}
\newcommand{\CharTok}[1]{\textcolor[rgb]{0.31,0.60,0.02}{#1}}
\newcommand{\CommentTok}[1]{\textcolor[rgb]{0.56,0.35,0.01}{\textit{#1}}}
\newcommand{\CommentVarTok}[1]{\textcolor[rgb]{0.56,0.35,0.01}{\textbf{\textit{#1}}}}
\newcommand{\ConstantTok}[1]{\textcolor[rgb]{0.00,0.00,0.00}{#1}}
\newcommand{\ControlFlowTok}[1]{\textcolor[rgb]{0.13,0.29,0.53}{\textbf{#1}}}
\newcommand{\DataTypeTok}[1]{\textcolor[rgb]{0.13,0.29,0.53}{#1}}
\newcommand{\DecValTok}[1]{\textcolor[rgb]{0.00,0.00,0.81}{#1}}
\newcommand{\DocumentationTok}[1]{\textcolor[rgb]{0.56,0.35,0.01}{\textbf{\textit{#1}}}}
\newcommand{\ErrorTok}[1]{\textcolor[rgb]{0.64,0.00,0.00}{\textbf{#1}}}
\newcommand{\ExtensionTok}[1]{#1}
\newcommand{\FloatTok}[1]{\textcolor[rgb]{0.00,0.00,0.81}{#1}}
\newcommand{\FunctionTok}[1]{\textcolor[rgb]{0.00,0.00,0.00}{#1}}
\newcommand{\ImportTok}[1]{#1}
\newcommand{\InformationTok}[1]{\textcolor[rgb]{0.56,0.35,0.01}{\textbf{\textit{#1}}}}
\newcommand{\KeywordTok}[1]{\textcolor[rgb]{0.13,0.29,0.53}{\textbf{#1}}}
\newcommand{\NormalTok}[1]{#1}
\newcommand{\OperatorTok}[1]{\textcolor[rgb]{0.81,0.36,0.00}{\textbf{#1}}}
\newcommand{\OtherTok}[1]{\textcolor[rgb]{0.56,0.35,0.01}{#1}}
\newcommand{\PreprocessorTok}[1]{\textcolor[rgb]{0.56,0.35,0.01}{\textit{#1}}}
\newcommand{\RegionMarkerTok}[1]{#1}
\newcommand{\SpecialCharTok}[1]{\textcolor[rgb]{0.00,0.00,0.00}{#1}}
\newcommand{\SpecialStringTok}[1]{\textcolor[rgb]{0.31,0.60,0.02}{#1}}
\newcommand{\StringTok}[1]{\textcolor[rgb]{0.31,0.60,0.02}{#1}}
\newcommand{\VariableTok}[1]{\textcolor[rgb]{0.00,0.00,0.00}{#1}}
\newcommand{\VerbatimStringTok}[1]{\textcolor[rgb]{0.31,0.60,0.02}{#1}}
\newcommand{\WarningTok}[1]{\textcolor[rgb]{0.56,0.35,0.01}{\textbf{\textit{#1}}}}
\usepackage{graphicx}
\makeatletter
\def\maxwidth{\ifdim\Gin@nat@width>\linewidth\linewidth\else\Gin@nat@width\fi}
\def\maxheight{\ifdim\Gin@nat@height>\textheight\textheight\else\Gin@nat@height\fi}
\makeatother
% Scale images if necessary, so that they will not overflow the page
% margins by default, and it is still possible to overwrite the defaults
% using explicit options in \includegraphics[width, height, ...]{}
\setkeys{Gin}{width=\maxwidth,height=\maxheight,keepaspectratio}
% Set default figure placement to htbp
\makeatletter
\def\fps@figure{htbp}
\makeatother
\setlength{\emergencystretch}{3em} % prevent overfull lines
\providecommand{\tightlist}{%
  \setlength{\itemsep}{0pt}\setlength{\parskip}{0pt}}
\setcounter{secnumdepth}{-\maxdimen} % remove section numbering
\ifLuaTeX
  \usepackage{selnolig}  % disable illegal ligatures
\fi
\IfFileExists{bookmark.sty}{\usepackage{bookmark}}{\usepackage{hyperref}}
\IfFileExists{xurl.sty}{\usepackage{xurl}}{} % add URL line breaks if available
\urlstyle{same} % disable monospaced font for URLs
\hypersetup{
  pdftitle={Worksheet3b\_Cartoja},
  pdfauthor={Jhona Mae Cartoja},
  hidelinks,
  pdfcreator={LaTeX via pandoc}}

\title{Worksheet3b\_Cartoja}
\author{Jhona Mae Cartoja}
\date{2022-11-21}

\begin{document}
\maketitle

\#1. Create a data frame using the table below. \#a. Write the codes.

\begin{Shaded}
\begin{Highlighting}[]
\NormalTok{Respondents }\OtherTok{\textless{}{-}} \FunctionTok{c}\NormalTok{(}\FunctionTok{seq}\NormalTok{(}\DecValTok{1}\NormalTok{,}\DecValTok{20}\NormalTok{))}
\NormalTok{Sex }\OtherTok{\textless{}{-}} \FunctionTok{c}\NormalTok{(}\DecValTok{2}\NormalTok{,}\DecValTok{2}\NormalTok{,}\DecValTok{1}\NormalTok{,}\DecValTok{2}\NormalTok{,}\DecValTok{2}\NormalTok{,}\DecValTok{2}\NormalTok{,}\DecValTok{2}\NormalTok{,}\DecValTok{2}\NormalTok{,}\DecValTok{2}\NormalTok{,}\DecValTok{2}\NormalTok{,}\DecValTok{1}\NormalTok{,}\DecValTok{2}\NormalTok{,}\DecValTok{2}\NormalTok{,}\DecValTok{2}\NormalTok{,}\DecValTok{2}\NormalTok{,}\DecValTok{2}\NormalTok{,}\DecValTok{2}\NormalTok{,}\DecValTok{2}\NormalTok{,}\DecValTok{1}\NormalTok{,}\DecValTok{2}\NormalTok{)}
\NormalTok{FathersOccupation }\OtherTok{\textless{}{-}} \FunctionTok{c}\NormalTok{(}\DecValTok{1}\NormalTok{,}\DecValTok{3}\NormalTok{,}\DecValTok{3}\NormalTok{,}\DecValTok{3}\NormalTok{,}\DecValTok{1}\NormalTok{,}\DecValTok{2}\NormalTok{,}\DecValTok{3}\NormalTok{,}\DecValTok{1}\NormalTok{,}\DecValTok{1}\NormalTok{,}\DecValTok{1}\NormalTok{,}\DecValTok{3}\NormalTok{,}\DecValTok{2}\NormalTok{,}\DecValTok{1}\NormalTok{,}\DecValTok{3}\NormalTok{,}\DecValTok{3}\NormalTok{,}\DecValTok{1}\NormalTok{,}\DecValTok{3}\NormalTok{,}\DecValTok{1}\NormalTok{,}\DecValTok{2}\NormalTok{,}\DecValTok{1}\NormalTok{)}
\NormalTok{Personsathome }\OtherTok{\textless{}{-}} \FunctionTok{c}\NormalTok{(}\DecValTok{5}\NormalTok{,}\DecValTok{7}\NormalTok{,}\DecValTok{3}\NormalTok{,}\DecValTok{8}\NormalTok{,}\DecValTok{5}\NormalTok{,}\DecValTok{9}\NormalTok{,}\DecValTok{6}\NormalTok{,}\DecValTok{7}\NormalTok{,}\DecValTok{8}\NormalTok{,}\DecValTok{4}\NormalTok{,}\DecValTok{7}\NormalTok{,}\DecValTok{5}\NormalTok{,}\DecValTok{4}\NormalTok{,}\DecValTok{7}\NormalTok{,}\DecValTok{8}\NormalTok{,}\DecValTok{8}\NormalTok{,}\DecValTok{3}\NormalTok{,}\DecValTok{11}\NormalTok{,}\DecValTok{7}\NormalTok{,}\DecValTok{6}\NormalTok{) }
\NormalTok{Siblingsatschool }\OtherTok{\textless{}{-}} \FunctionTok{c}\NormalTok{(}\DecValTok{6}\NormalTok{,}\DecValTok{4}\NormalTok{,}\DecValTok{4}\NormalTok{,}\DecValTok{1}\NormalTok{,}\DecValTok{2}\NormalTok{,}\DecValTok{1}\NormalTok{,}\DecValTok{5}\NormalTok{,}\DecValTok{3}\NormalTok{,}\DecValTok{1}\NormalTok{,}\DecValTok{2}\NormalTok{,}\DecValTok{3}\NormalTok{,}\DecValTok{2}\NormalTok{,}\DecValTok{5}\NormalTok{,}\DecValTok{5}\NormalTok{,}\DecValTok{2}\NormalTok{,}\DecValTok{1}\NormalTok{,}\DecValTok{2}\NormalTok{,}\DecValTok{5}\NormalTok{,}\DecValTok{3}\NormalTok{,}\DecValTok{2}\NormalTok{)}
\NormalTok{Typesofhouses }\OtherTok{\textless{}{-}} \FunctionTok{c}\NormalTok{(}\DecValTok{1}\NormalTok{,}\DecValTok{2}\NormalTok{,}\DecValTok{3}\NormalTok{,}\DecValTok{1}\NormalTok{,}\DecValTok{1}\NormalTok{,}\DecValTok{3}\NormalTok{,}\DecValTok{3}\NormalTok{,}\DecValTok{1}\NormalTok{,}\DecValTok{2}\NormalTok{,}\DecValTok{3}\NormalTok{,}\DecValTok{2}\NormalTok{,}\DecValTok{3}\NormalTok{,}\DecValTok{2}\NormalTok{,}\DecValTok{2}\NormalTok{,}\DecValTok{3}\NormalTok{,}\DecValTok{3}\NormalTok{,}\DecValTok{3}\NormalTok{,}\DecValTok{3}\NormalTok{,}\DecValTok{3}\NormalTok{,}\DecValTok{2}\NormalTok{)}

\NormalTok{dataframe }\OtherTok{\textless{}{-}} \FunctionTok{data.frame}\NormalTok{(Respondents,Sex,FathersOccupation,Personsathome,Siblingsatschool,Typesofhouses)}
\end{Highlighting}
\end{Shaded}

\#b.Describe the data. Get the structure or the summary of the data

\begin{Shaded}
\begin{Highlighting}[]
\FunctionTok{summary}\NormalTok{(dataframe)}
\end{Highlighting}
\end{Shaded}

\begin{verbatim}
##   Respondents         Sex       FathersOccupation Personsathome 
##  Min.   : 1.00   Min.   :1.00   Min.   :1.00      Min.   : 3.0  
##  1st Qu.: 5.75   1st Qu.:2.00   1st Qu.:1.00      1st Qu.: 5.0  
##  Median :10.50   Median :2.00   Median :2.00      Median : 7.0  
##  Mean   :10.50   Mean   :1.85   Mean   :1.95      Mean   : 6.4  
##  3rd Qu.:15.25   3rd Qu.:2.00   3rd Qu.:3.00      3rd Qu.: 8.0  
##  Max.   :20.00   Max.   :2.00   Max.   :3.00      Max.   :11.0  
##  Siblingsatschool Typesofhouses
##  Min.   :1.00     Min.   :1.0  
##  1st Qu.:2.00     1st Qu.:2.0  
##  Median :2.50     Median :2.5  
##  Mean   :2.95     Mean   :2.3  
##  3rd Qu.:4.25     3rd Qu.:3.0  
##  Max.   :6.00     Max.   :3.0
\end{verbatim}

\#c.~Is the mean number of siblings attending is 5?

Ans: No

\#d.~Extract the 1st two rows and then all the columns using the
subsetting functions. Write the codes and its output.

\begin{Shaded}
\begin{Highlighting}[]
\NormalTok{A1 }\OtherTok{\textless{}{-}} \FunctionTok{subset}\NormalTok{(dataframe[}\DecValTok{1}\SpecialCharTok{:}\DecValTok{2}\NormalTok{, }\DecValTok{1}\SpecialCharTok{:}\DecValTok{6}\NormalTok{, }\AttributeTok{drop =} \ConstantTok{FALSE}\NormalTok{])}
\NormalTok{A1}
\end{Highlighting}
\end{Shaded}

\begin{verbatim}
##   Respondents Sex FathersOccupation Personsathome Siblingsatschool
## 1           1   2                 1             5                6
## 2           2   2                 3             7                4
##   Typesofhouses
## 1             1
## 2             2
\end{verbatim}

\#e. Extract 3rd and 5th row with 2nd and 4th column. Write the codes
and its \#result.

\begin{Shaded}
\begin{Highlighting}[]
\NormalTok{A2 }\OtherTok{\textless{}{-}} \FunctionTok{subset}\NormalTok{(dataframe[}\FunctionTok{c}\NormalTok{(}\DecValTok{3}\NormalTok{,}\DecValTok{5}\NormalTok{),}\FunctionTok{c}\NormalTok{(}\DecValTok{2}\NormalTok{,}\DecValTok{4}\NormalTok{)])}
\NormalTok{A2}
\end{Highlighting}
\end{Shaded}

\begin{verbatim}
##   Sex Personsathome
## 3   1             3
## 5   2             5
\end{verbatim}

\#f.~Select the variable types of houses then store the vector that
results as types\_houses. Write the codes.

\begin{Shaded}
\begin{Highlighting}[]
\NormalTok{A3 }\OtherTok{\textless{}{-}} \FunctionTok{subset}\NormalTok{(dataframe[}\FunctionTok{c}\NormalTok{(}\DecValTok{1}\SpecialCharTok{:}\DecValTok{20}\NormalTok{), }\FunctionTok{c}\NormalTok{(}\DecValTok{2}\NormalTok{,}\DecValTok{6}\NormalTok{)])}
\NormalTok{type\_houses }\OtherTok{\textless{}{-}}\NormalTok{ A3}
\end{Highlighting}
\end{Shaded}

\#g. Select only all Males respondent that their father occupation was
farmer. Write the codes and its output.

\begin{Shaded}
\begin{Highlighting}[]
\NormalTok{A4 }\OtherTok{\textless{}{-}} \FunctionTok{subset}\NormalTok{(dataframe[}\FunctionTok{c}\NormalTok{(}\DecValTok{1}\SpecialCharTok{:}\DecValTok{20}\NormalTok{), }\FunctionTok{c}\NormalTok{(}\DecValTok{2}\NormalTok{,}\DecValTok{3}\NormalTok{)])}
\NormalTok{Males }\OtherTok{\textless{}{-}}\NormalTok{ A4[dataframe}\SpecialCharTok{$}\NormalTok{FathersOccupation }\SpecialCharTok{==} \StringTok{\textquotesingle{}1\textquotesingle{}}\NormalTok{,]}
\NormalTok{Males}
\end{Highlighting}
\end{Shaded}

\begin{verbatim}
##    Sex FathersOccupation
## 1    2                 1
## 5    2                 1
## 8    2                 1
## 9    2                 1
## 10   2                 1
## 13   2                 1
## 16   2                 1
## 18   2                 1
## 20   2                 1
\end{verbatim}

\#h. Select only all females respondent that have greater than or equal
to 5 number \#of siblings attending school. Write the codes and its
outputs

\begin{Shaded}
\begin{Highlighting}[]
\NormalTok{A5 }\OtherTok{\textless{}{-}} \FunctionTok{subset}\NormalTok{(dataframe[}\FunctionTok{c}\NormalTok{(}\DecValTok{1}\SpecialCharTok{:}\DecValTok{20}\NormalTok{), }\FunctionTok{c}\NormalTok{(}\DecValTok{2}\NormalTok{,}\DecValTok{5}\NormalTok{)])}
\NormalTok{females }\OtherTok{\textless{}{-}}\NormalTok{ A5[dataframe}\SpecialCharTok{$}\NormalTok{Siblingsatschool }\SpecialCharTok{==} \StringTok{\textquotesingle{}1\textquotesingle{}}\NormalTok{,]}
\NormalTok{females}
\end{Highlighting}
\end{Shaded}

\begin{verbatim}
##    Sex Siblingsatschool
## 4    2                1
## 6    2                1
## 9    2                1
## 16   2                1
\end{verbatim}

\#2. Write a R program to create an empty data frame. Using the
following codes:

\begin{Shaded}
\begin{Highlighting}[]
\NormalTok{dataframe }\OtherTok{=} \FunctionTok{data.frame}\NormalTok{(}\AttributeTok{Integers=}\FunctionTok{integer}\NormalTok{(),}
                \AttributeTok{Doubles=}\FunctionTok{double}\NormalTok{(), }
                \AttributeTok{Characters=}\FunctionTok{character}\NormalTok{(),}
                \AttributeTok{Logicals=}\FunctionTok{logical}\NormalTok{(),}
                \AttributeTok{Factors=}\FunctionTok{factor}\NormalTok{(),}
                \AttributeTok{stringsAsFactors=}\ConstantTok{FALSE}\NormalTok{)}

          \FunctionTok{print}\NormalTok{(}\StringTok{"Structure of the empty dataframe:"}\NormalTok{)}
\end{Highlighting}
\end{Shaded}

\begin{verbatim}
## [1] "Structure of the empty dataframe:"
\end{verbatim}

\begin{Shaded}
\begin{Highlighting}[]
          \FunctionTok{print}\NormalTok{(}\FunctionTok{str}\NormalTok{(dataframe))}
\end{Highlighting}
\end{Shaded}

\begin{verbatim}
## 'data.frame':    0 obs. of  5 variables:
##  $ Integers  : int 
##  $ Doubles   : num 
##  $ Characters: chr 
##  $ Logicals  : logi 
##  $ Factors   : Factor w/ 0 levels: 
## NULL
\end{verbatim}

\end{document}
